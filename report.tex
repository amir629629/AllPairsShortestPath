\documentclass[twocolumn]{article}
\usepackage{amsmath}
\usepackage{graphicx}
\graphicspath{{Images/}}
\usepackage{cleveref}

\title{CS141 -- Intermediate Algorithms and Data Structures\\Assignment 2 -- All
Pairs Shortest Path}
\author{Anthony Martinez}
\date{5/27/15}

\begin{document}
\maketitle

\section*{Abstract}
\textbf{\small We will be comparing the run time of two separate algorithms for determing the shortest distance between
all pairs of vertices in a graph. The first algorithm is Bellman-Ford's, which was originally developed for
finding the shortest path between a single source vertex and all other vertices. We will extend this algorithm to find
all pairs shortest paths. The next algorithm is Floyd-Warshall's, which was created to solve this problem faster than
the extended Bellman-Ford algorithm.}

\section{Assignment}
\label{sec::assignment}
\begin{itemize}
    \item Implement Bellman-Ford's algorithm to find the length of the path between a source (s) and all other vertices.
    \begin{itemize}
        \item Extend Bellman-Ford's to find the length of the paths between all pairs of vertices.
    \end{itemize}
    \item Implement Floyd-Warshall's algorithm to find the length of the paths between all pairs of vertices.
    \item Calculate the run-time of all of the implemented algorithms.
    \item Run all of the benchmarks on both "all pairs shortest path" algorithms.
    \item Fill in the report analyzing the algorithms and their run-time.
    \item You may remove the Assignment section (\cref{sec::assignment}) before turning in the final report
\end{itemize}

\section{Introduction}
\begin{itemize}
    \item What is the problem that you are solving?
    \item What methods are you going to use to solve the problem?
    \item Why are these good methods to use?
    \item Why are you going to be using both of them?
\end{itemize}

\section{Bellman-Ford}
\begin{itemize}
    \item What is the Bellman-Ford algorithm?
    \item Why are you using it?
    \item How did you adapt it to work for all-pairs as opposed to single
    source?
    \item What is the run-time of the algorithm before and after your
    adaptation?
\end{itemize}

\section{Floyd-Warshall}
\begin{itemize}
    \item What is the Floyd-Warshall algorithm?
    \item Why are you using it?
    \item How is it better than the Bellman-Ford algorithm?
    \item What is the run-time of the algorithm?
\end{itemize}

\section{Results}
\begin{itemize}
    \item Compare and contrast the two algorithms? What makes one more suited
    for this problem?
    \item What are their theoretical run-times (from the previous sections) and
    how do they compare?
    \item What are the actual run-times that you computed? Which method is
    better? Why?
    \item Fill in \cref{table::run-time} with your results
\end{itemize}

% Fill in the following table
\begin{table}
    \caption{Run-Time Comparison}
    \begin{tabular}{|c|c|c|c|}
        \hline
        &\#&Bellman-Ford&Floyd-Warshall\\
        Benchmarks &Edges&Actual&Actual\\
        \hline
        input4.txt   &   5& 0.000111&      \\
        input5.txt   &   8& 0.000203&      \\
        input10.txt  &  16& 0.00158&      \\
        input25.txt  &  43& 0.024964&      \\
        input50.txt  &  93& 0.208175&      \\
        input100.txt & 192& 1.857163&      \\
        input250.txt & 730& 41.177519&      \\
        input500.txt &1532& 339.405867&      \\
        input1000.txt&2985& 2633.958733&      \\
        \hline
    \end{tabular}
    \label{table::run-time}
\end{table}

\section{Conclusions}
\begin{itemize}
    \item What did you find difficult about the assignment?
    \item What did you learn?
    \item What is one real-world problem that you think each of these problems
    would be good at solving?
\end{itemize}

\end{document}
